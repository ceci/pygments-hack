\documentclass[a4paper, 11pt, oneside]{book}

%% For portuguese compatibility
\usepackage[brazil]{babel}
\usepackage[latin1]{inputenc}
\usepackage[T1]{fontenc}
\usepackage{float}
\usepackage[scaled]{helvet}
\renewcommand*\familydefault{\sfdefault}
\usepackage{indentfirst}

%%%%%%%%%%%%%%%%%%%%%%%%%%%%%%%%%%%%%%%%%%%%%%%%%%%%%%%%%%%%%%%%%%%%%%
%% Packages
\usepackage[dvips]{graphicx,psfrag} 
\usepackage{ifthen}
\usepackage{makeidx}
\usepackage{enumerate}
\usepackage{fancyhdr}
\usepackage{minted}
\usepackage{pstcol}
\usepackage{colortbl}
\usepackage{url}  
\usepackage{rotating}
\usepackage{multirow}
\usepackage{colortbl}
\usepackage[table]{xcolor}

\usepackage{tubaina}

%This has to be the last package declared, for some reason
\usepackage[pdftex]{hyperref}

\definecolor{boxhead}{gray}{0.8}

%%%%%%%PAGE SETUP%%%%%%%%%%%%%
\oddsidemargin    -6mm
\textwidth        180mm
\topmargin        -10mm
\textheight       250mm
%%%%%%%%%%%%%%%%%%%%%%%%%%%%%%

\makeatother

\begin{document}


\section{Nova tubainabox}

Uma tentativa agora com um environment pra criar boxes.

\begin{tubainabox}{Uma note qualquer}
�s vezes a gente quer escrever uma note extensa e que tem c�digo dentro, mas o minted zuava 
todo o c�digo, ent�o agora estamos usando minipage em vez de parbox e parece que resolvemos 
o problema.

�s vezes a gente quer escrever uma note extensa e que tem c�digo dentro, mas o minted zuava 
todo o c�digo, ent�o agora estamos usando minipage em vez de parbox e parece que resolvemos 
o problema.}

{\small
  \begin{minted}[linenos]{ruby}
    def menu_principal
    menu = %w(cliente qualificac�o restaurante)
    menu_principal = "<ul>"
    menu.each do |item|
    menu_principal << "<li>" + link_to(item, :controller => item.pluralize) + "</li>"
    end
    menu_principal << "</ul>"
    raw menu_principal
    end
  \end{minted}
\end{tubainabox}


\section{teste com linguagem}


{\small
  \begin{minted}[linenos, numbersep:5px]{java}
    @Entity
    public class Card implements Identifiable, FromProject {

      @Id
      @GeneratedValue
      private Long id;

      @Enumerated(EnumType.STRING)
      private Status status = Status.TODO;

      public static enum Status {
	TODO, DONE
      }

      private CardRepository getRepository() {
	if (repository == null) {
	  throw new IllegalStateException("Repository was not set. You should" +  
                                                              "inject it first");
	}
	return repository;
      }

      public List<Long> deleteSubCards() {
	List<Long> ids = new ArrayList<Long>();
	for (Card sub : getSubcards()) {
	  ids.add(sub.getId());
          getRepository().remove(sub);
        }
	return ids;
      }
    }  
  \end{minted}
}

\section{Trechos do FJ-11}

\\[1em]
{\small
\begin{minted}[mathescape]{text}
javap -c MeuPrograma
\end{minted}
}
\\[1em]

E a sa�da:
\\[1em]

{\small
\begin{minted}{text}
MeuPrograma();
  Code:
   0:   aload_0
   1:   invokespecial   #1; //Method java/lang/Object.''<init>'':()V
   4:   return

public static void main(java.lang.String[]);
  Code:
   0:   getstatic       #2; //Field java/lang/System.out:Ljava/io/PrintStream;
   3:   ldc     #3; //String Minha primeira aplicacao Java!!
   5:   invokevirtual   #4; //Method java/io/PrintStream.println:(Ljava/lang/String;)
   8:   return
\end{minted}
}
\\[1em]


\end{document}
